%This is the default Latex template from Pandoc used for pdf output. I added the pandocbounded comand in line 11 to fix an error. If you need to adjust parameters like margins, then put that in the frontmatter, that is, the metadata.yaml file.
% Options for packages loaded elsewhere
\PassOptionsToPackage{unicode}{hyperref}
\PassOptionsToPackage{hyphens}{url}
\newcommand{\pandocbounded}[1]{#1}
%
\documentclass[
]{article}
\usepackage{amsmath,amssymb}
\usepackage{lmodern}
\usepackage{iftex}
\ifPDFTeX
  \usepackage[T1]{fontenc}
  \usepackage[utf8]{inputenc}
  \usepackage{textcomp} % provide euro and other symbols
\else % if luatex or xetex
  \usepackage{unicode-math}
  \defaultfontfeatures{Scale=MatchLowercase}
  \defaultfontfeatures[\rmfamily]{Ligatures=TeX,Scale=1}
\fi
% Use upquote if available, for straight quotes in verbatim environments
\IfFileExists{upquote.sty}{\usepackage{upquote}}{}
\IfFileExists{microtype.sty}{% use microtype if available
  \usepackage[]{microtype}
  \UseMicrotypeSet[protrusion]{basicmath} % disable protrusion for tt fonts
}{}
\makeatletter
\@ifundefined{KOMAClassName}{% if non-KOMA class
  \IfFileExists{parskip.sty}{%
    \usepackage{parskip}
  }{% else
    \setlength{\parindent}{0pt}
    \setlength{\parskip}{6pt plus 2pt minus 1pt}}
}{% if KOMA class
  \KOMAoptions{parskip=half}}
\makeatother
\usepackage{xcolor}
\usepackage[left=2.5cm, right=2.5cm]{geometry}
\setlength{\emergencystretch}{3em} % prevent overfull lines
\providecommand{\tightlist}{%
  \setlength{\itemsep}{0pt}\setlength{\parskip}{0pt}}
\setcounter{secnumdepth}{-\maxdimen} % remove section numbering
\newlength{\cslhangindent}
\setlength{\cslhangindent}{1.5em}
\newlength{\csllabelwidth}
\setlength{\csllabelwidth}{3em}
\newlength{\cslentryspacingunit} % times entry-spacing
\setlength{\cslentryspacingunit}{\parskip}
\newenvironment{CSLReferences}[2] % #1 hanging-ident, #2 entry spacing
 {% don't indent paragraphs
  \setlength{\parindent}{0pt}
  % turn on hanging indent if param 1 is 1
  \ifodd #1
  \let\oldpar\par
  \def\par{\hangindent=\cslhangindent\oldpar}
  \fi
  % set entry spacing
  \setlength{\parskip}{#2\cslentryspacingunit}
 }%
 {}
\usepackage{calc}
\newcommand{\CSLBlock}[1]{#1\hfill\break}
\newcommand{\CSLLeftMargin}[1]{\parbox[t]{\csllabelwidth}{#1}}
\newcommand{\CSLRightInline}[1]{\parbox[t]{\linewidth - \csllabelwidth}{#1}\break}
\newcommand{\CSLIndent}[1]{\hspace{\cslhangindent}#1}
\ifLuaTeX
\usepackage[bidi=basic]{babel}
\else
\usepackage[bidi=default]{babel}
\fi
\babelprovide[main,import]{english}
% get rid of language-specific shorthands (see #6817):
\let\LanguageShortHands\languageshorthands
\def\languageshorthands#1{}
\ifLuaTeX
  \usepackage{selnolig}  % disable illegal ligatures
\fi
\IfFileExists{bookmark.sty}{\usepackage{bookmark}}{\usepackage{hyperref}}
\IfFileExists{xurl.sty}{\usepackage{xurl}}{} % add URL line breaks if available
\urlstyle{same} % disable monospaced font for URLs
\hypersetup{
  pdftitle={The Relation Between Content Classification and UX Aspects in Documentation Portals},
  pdflang={en},
  pdfkeywords={technical documentation, user eXperience, content
delivery portals, documentation portals, classification},
  hidelinks,
  pdfcreator={LaTeX via pandoc}}

\title{The Relation Between Content Classification and UX Aspects in
Documentation Portals}
\author{true}
\date{}

\begin{document}
\maketitle
\begin{abstract}
Documentation portals that deliver technical documentation of multiple
products to multiple user groups carry the company's brand and serve
business goals. The users' experience in such portals is therefore an
important factor. Findability and context provision typically impact
user experience in content-heavy applications. Two content
characteristics that can influence these UX-relevant factors are
classification and modularization of content. In the past, technical
writers handled classification and modularization of technical
documentation when preparing technical documentation for delivery.
However, in the past decade, processes that typically have been employed
to use classification and modularization to ensure findability and
context provision in technical documentation were extended by additional
technologies or have changed entirely.

Since documentation portals become widespread and integrate more sources
and users, this paper investigates the research question: What is the
relationship between classification and modularization of content on
documentation creation on the one hand and the user experience in a
documentation portal on the other hand? To this end, expert interviews
with software producers were conducted, and documentary sources
reviewed.

The relationship between user experience in documentation portals and
content classification and modularization was found to be moderated by
use cases. With the relationship clarified, issues that surface in
usability tests and user experience research can be addressed in a more
targeted way. The relationship can help understand where in a specific
process the user experience can be influenced and by whom.
\end{abstract}

\renewcommand*\contentsname{Contents}
{
\setcounter{tocdepth}{3}
\tableofcontents
}
\subsection{Introduction}\label{introduction}

\subsubsection{User Experience in Documentation
Portals}\label{user-experience-in-documentation-portals}

For companies, it has become convenient to publish technical
documentation for several products together with other product related
information in one documentation portal to serve several user groups
(Antidot, 2020; Ziegler \& Beier, 2014). Documentation portals carry the
company's brand. The experience users have in such a portal directly
connects back to the brand (Lee et al., 2018).

\subsection*{Bibliography}\label{bibliography}
\addcontentsline{toc}{subsection}{Bibliography}

\protect\phantomsection\label{refs}
\begin{CSLReferences}{1}{0}
\bibitem[\citeproctext]{ref-Antidot2020}
Antidot. (2020). \emph{Dynamic delivery. What is it and why does it
matter.}
https://www.fluidtopics.com/wp-content/uploads/sites/3/FluidTopics-WP-DynamicDelivery-202011en.pdf.
\CSLBlock{Whitepaper. Accessed December 22.}

\bibitem[\citeproctext]{ref-LeeChoi2018}
Lee, H., Lee, K. K., \& Choi, J. (2018). A structural model for unity of
experience: Connecting user experience, customer experience, and brand
experience. \emph{Journal of Usability Studies}, \emph{14}, 8--13.
\url{https://uxpajournal.org/wp-content/uploads/sites/7/pdf/JUS_Lee_Nov2018.pdf}

\bibitem[\citeproctext]{ref-ZieglerBeier2014}
Ziegler, W., \& Beier, H. (2014). \emph{Alles muss raus}.
\CSLBlock{technische kommunikation, H. 6, S. 50-55. Accessed December
22.}

\end{CSLReferences}

\end{document}
